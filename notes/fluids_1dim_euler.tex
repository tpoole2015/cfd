\documentclass[12pt]{article}    

\usepackage{amsmath,amsfonts,amssymb}   %% AMS mathematics macros

%% Title Information.

\title{Derivation of the quasi 1-dim Euler equations}
\author{Thomas Poole}

%%%%% The Document

\begin{document}

\maketitle

Our starting point is Reynold's transport theorem
\begin{equation}
\label{reynolds}
	\frac{dN}{dt} = \frac{\partial}{\partial t}\int_{V}\eta\rho d\Omega + \int_{S}\eta\rho U\cdot dA
\end{equation}
where $N$ is an extensive property associated with the control volume $V$ and $\eta$ is the corresponding intensive property. That is
$$
 N(t) = \int_{V}\eta(x,t)\rho(x,t) d\Omega 
$$
We're considering 1-dim flow and so $U(x,t)=u(x,t)\hat{x}$. We take our system to be the fluid contained in the control volume $V$ at time $t = 0$. It's important to note that for $t > 0$ the control volume is most likely no longer our system! (our system has flowed in the $\hat{x}$ direction).

\section{Conservation of mass}
Setting $\eta = 1$ in \eqref{reynolds} gives us $N = $mass system. As the mass is not changing we get
$$
 0 = \frac{dN}{dt} =  \frac{\partial}{\partial t}\int_{V}\rho d\Omega + \int_{S}\rho U\cdot dA.
$$
If we now take our control volume $V$ to be a cylinder from $x$ to $x + dx$ where each cross section has area $A(x)$ we get
$$ 
\frac{\partial}{\partial t}(\rho Adx)(x,t) + (\rho uA)(x+dx,t) - (\rho uA)(x,t) = 0 
$$
and so
\begin{equation}
\label{consmass}
\boxed{\frac{\partial}{\partial t}(\rho A)(x,t) +  \frac{\partial}{\partial x}(\rho uA)(x,t) = 0}
\end{equation}
The equation \eqref{consmass} is the differential equation for the conservation of mass.

\section{Conservation of momentum}
Letting $N = $ momentum of our system (associated with the control volume at time $t=0$) in the $\hat{x}$ direction, then $\eta=\rho u$. Therefore
$$
 0 = \frac{dN}{dt} = \frac{\partial}{\partial t}\int_{V}\rho ud\Omega + \int_{S}\rho uU\cdot dA.
$$
By Newton's second law we have
\begin{align*}
 \frac{dN}{dt} &= \sum_{F \text{acting on system}}F_x \\
                        &= F^{\text{surface}}_x + F^{\text{body}}_x.
\end{align*}
Ignoring the body forces and just considering the surface forces (we're also neglecting shear forces, i.e. viscosity) gives us
\begin{align*}
 F^{\text{surface}}_x &= \text{pressure on left side of CV} + \text{pressure on right side of CV} + \text{pressure on area of CV connecting left and right sides} \\
 &= (\rho A)(x,t) - (\rho A)(x+dx,t) + (p + \frac{dp}{2})dA(x,t)
\end{align*}
where we now have the fluid pressure $p = p(x,t)$ (force per unit area exerted by our fluid) appear in our equation. Combining the above gives us
$$
\frac{\partial}{\partial t}(\rho uAdx)(x,t) + (\rho u^2A)(x+dx,t) - (\rho u^2A)(x,t) = (pA)(x,t) - (pA)(x+dx,t) + (p+\frac{dp}{2})dA(x,t) 
$$
and therefore
\begin{equation}
\label{consmom}
\boxed{\frac{\partial}{\partial t}(\rho uA) +  \frac{\partial}{\partial x}[(\rho u^2+p)A] = p\frac{dA}{dx}}
\end{equation}
which is the differential equation for the conservation of momentum.

\section{Conservation of energy}
Letting $N = E = $ energy of the system, then $\eta = \epsilon = $ energy per unit mass. So by \eqref{reynolds} we get
\begin{equation}
\label{coneng-reynolds}
\frac{dE}{dt} = \frac{\partial}{\partial t}\int_{V}\rho \epsilon d\Omega + \int_{S}\rho \epsilon U\cdot dA
                       = \frac{\partial}{\partial t}\int_{V}ed\Omega + \int_{S}eU\cdot dA
\end{equation}
where $e = \epsilon\rho$ is the energy per unit volume. From thermodynamics we have
\begin{align*}
\frac{dE}{dt} &= (\text{heat added to the system}) - (\text{rate at which work is done by the system}) \\
                       &= \frac{dQ}{dt} - \frac{dW}{dt}
\end{align*}
we'll assume our system is adiabatic and so ignore $dQ/dt$. Consider a small unit of area of our control surface $dA$, then the work done by the system on $dA$ (displacing it by $dx$) is
$$
F\cdot dx = pdA\cdot dx
$$
and so the rate at which work is done by the system on $dA$ is
$$
pdA\cdot U
$$
therefore
$$
\frac{dW}{dt} = \int_SpU\cdot dA.
$$
If we plug this into ~\eqref{coneng-reynolds} we get
$$
\frac{\partial}{\partial t}\int_{V}ed\Omega + \int_{S}eU\cdot dA = -\int_S{pU\cdot dA}
$$
and so applying it to our cylindrical control volume gives us
$$
\frac{\partial}{\partial t}(eAdx)(x,t) + (euA)(x+dx,t) - (euA)(x,t) = -(puA)(x+dx,t) + (puA)(x,t)
$$
dividing by $dx$ and taking the limit gives us the PDE for conservation of energy
\begin{equation}
\boxed{
\label{conseng}
\frac{\partial}{\partial t}(eA) +  \frac{\partial}{\partial x}[(e+p)uA] = 0
}
\end{equation}

\section{Discussion}
We have derived 3 equations ~\eqref{consmass},~\eqref{consmom} and ~\eqref{conseng} in 4 unknowns $\rho,u,e$ and $p$. To close the system we use an equation of state. For example the ideal gas law gives us
$$
p=\rho(\gamma-1)e
$$
where $\gamma$ is the adiabatic index.
\end{document}
